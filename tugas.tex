\documentclass[12pt,A4paper]{article}
\usepackage[top=1cm,right=2cm,bottom=3cm,left=2cm]{geometry}
\usepackage{amsmath}
\usepackage{amssymb}
\begin{document}
	\begin{center}
		Muhammad Rokan Azriel Prananta
		\\ 24010121130045
		\\ Tugas $1$ Mata Kuliah Matematika Diskrit
		\\ Algoritma RSA
	\end{center}
	\begin{itemize}
		\item  $\mathbf{Definisi}$
		\\ RSA atau $Rivest-Shamir-Adleman$ merupakan algoritma kunci-publik yang ditemukan oleh Ronald Rivest, Adi Shamir, dan Leonard Adleman pada tahun $1976$. Keamanan algoritma RSA terletak pada kekompleksnya pengfaktoran bilangan bulat yang nilainya besar menjadi faktor-faktor bilangan prima.
		\item $\mathbf{Properti}$
		\begin{enumerate}
			\item $p$ dan $q$ sebagai bilangan prima
			\item $n = p \times q$
			\item  $\phi(n)= (p-1)(q-1)$
			\item $e$ sebagai kunci enkripsi dengan syarat $\gcd(e, \phi(n)) =1$
			\item $d$ sebagai kunci enkripsi di mana $d$ didapatkan dari perhitungan
			\[ d \equiv e^{-1} (\text{mod }\phi(n)) \]
			\item $m$ sebagai plainteks dan $c$ sebagai chiperteks
		\end{enumerate}
	\item $\mathbf{Rumus \: Enkripsi\:  dan \: Dekripsi}$
	\begin{enumerate}
		\item Enkripsi
		\[ E_e(m) = c = m^e (\text{mod } n)\]
		\item Dekripsi
		\[ D_d (c)= m = c^d (\text{mod } n) \]
	\end{enumerate}
	\item $\mathbf{Prosedur \: Pembangkita  \: Kunci}$
	\begin{enumerate}
		\item Memilih dua bilangan prima $p$ dan $q$
		\item Menghitung $n = p \times q$
		\item Menghitung $\phi(n)= (p-1)(q-1)$
		\item Memilih bilangan bulat $e$ sebagai kunci publik di mana $e$ harus relarif prima terhadap $\phi(n)$
		\item Menghitung kunci dekripsi $d$ dengan persamaan
		\[ed = 1 \left(\text{ mod } \phi(n)\right) \]
	\end{enumerate}
	Di mana hasil dari algoritma di atas adalah
	\begin{enumerate}
		\item Kunci publik adalah pasangan $(e, n)$
		\item Kunci privasi adalah pasangan $(d, n)$
	\end{enumerate}
\newpage
	\item $\mathbf{Langkah \: Enkripsi}$
	\begin{enumerate}
		\item Menyatakan pesan atau kode menjadi blok-blok chiperteks berupa
		\[m_1, m_2, \dots , m_n\]
		untuk $0 \leq m_1 < (n-1)$.
		\item Menghitung blok chiperteks $c_i$ untuk plainteks $m_i$ menggunakan kunci publik $e$ dengan persamaan
		\[c_i \equiv m_i^e (\text{mod }n)\]
	\end{enumerate}
	\item $\mathbf{Langkah \: Dekripsi}$
	\begin{enumerate}
		\item Memisalkan blok-blok chiperteks berupa
		\[c_1,c_2, \dots, c_n\]
		\item Menghitung kembali blok plainteks $m_i$ dari blok chiperteks $c_i$ menggunakan kunci privat $d$ dengan persamaan
		\[m_i \equiv c_i^d (\text{mod } n)\]
	\end{enumerate}
	\item $\mathbf{Contoh \: Pesan}$
	\begin{enumerate}
	\item Penyusunan Kunci-kunci pada Algoritma
	\\ Seorang peretas memilih dua buah bilangan prima,yaitu $p = 47$ dan $q=71$ sehingga dapat dikalkulasi nilai $n = 47 \times 71 = 3337$ dan $\phi(n) = 46 \times 70 = 3220$. Lalu, peretas tersebut memilih kunci publik $e = 79$ karena $\gcd(79, 3220) = 1$. Selanjutnya persetas melakukan kalkulasi kunci privat $d$ dengan rumus
	\[ed \equiv 1 (\text{mod }\phi(n))\]
	diperoleh nilai $d$ sebagai kebalikan dari nilai $e$ dalam modulus $\phi(n)$. Di mana nilai $d$ dapat dihitung dengan rumus
	\[d = \frac{1 + k \; \phi(n)}{e}\]
	Dengan melakukan substitusi nilai $k$ di mana $\forall k \in \mathbb{Z^+}$ sehingga diperolej nilai $d = 1019$ sebagai kunci privat untuk melakukan dekripsi.
	\item Pengiriman Pesan Kode (Enkripsi kode)
	\\ Berdasarkan nomor $1$, seorang kepala FBI mengirim plainteks kepada peretas berupa
	\[\text(HELLO \: KARIR \:ANDA \: SAMPAI \:DI \:SINI )\]
	Dengan memisalkan $A= 00, B=01, \dots, Z =25$, maka pesan di atas dikodekan ke dalam bilangan bulat dengan spasi yang diabaikan berupa 
	\[07 04 11 11 14 10 00 17 08 17 00 13 03 00 18 00 12 15 00 08 03 08 18 08 13 08 \]
	Memecahkan kode di atas menjadi blok yang setiap bloknya terdiri dari $4$ digit berupa
	\newpage
	\[m_1 = 0704\]
	\[m_2 = 1111 \]
	\[m_3 = 1410\]
	\[m_4 = 0017\]
	\[m_5 = 0817\]
	\[m_6 = 0013\]
	\[m_7 = 0300\]
	\[m_8 =1800\]
	\[m_9 = 1215\]
	\[m_{10} = 0008\]
	\[m_{11} = 0308\]
	\[m_{12} = 1808\]
	\[m_{13} = 1308\]
	Perhatikan bahwa nilai $m_i \in [0, 3337 -1]$.
	\\ Selanjutnya agen FBI melakukan enkripsi blok-blok di atas menggunakan kunci publik yang telah peretas tentukan $e = 79$ sedemikian sehingga diperoleh 
	\[c_1 = 704^{79} \text{mod } 3337 = 328\]
\[c_2 = 111^{79} \text{mod } 3337 = 301\]
\[c_3 = 1410^{79} \text{mod } 3337 = 423\]
\[c_4 = 17^{79} \text{mod } 3337 = 827\]
\[c_5 = 817^{79} \text{mod } 3337 = 729\]
\[c_6 = 13^{79} \text{mod } 3337 = 2066\]
\[c_7 = 300^{79} \text{mod } 3337 = 3126\]
\[c_8 = 1800^{79} \text{mod } 3337 = 338\]
\[c_9 = 1215^{79} \text{mod } 3337 = 109 \]
\[c_{10} = 8^{79} \text{mod } 3337 = 2807 \]
\[c_{11} = 308^{79} \text{mod } 3337 =537 \]
\[c_{12} = 1808^{79} \text{mod } 3337 =2112 \]
\[c_{13} = 1308^{79} \text{mod } 3337 = 1752\]
Dari perhitungan di atas diperoleh chiperteks 
\[0328 0301 0423 0827 0729 2066 3126 0338 0109 2807 0537 2112 1751\]
\item Dekripsi Kode 
\\ Setelah peretas menerima kode chiperteks yang telah dikirim oleh FBI, si peretas melakukan dekripsi pada chiperteks dengan menggunakan kunci privatnya $d = 1019$ menjadi
\[m_1 = 328^{1019} (\text{mod }) 3337 = 704\]
\[m_2 = 301^{1019} (\text{mod }) 3337 = 1111\]
\[m_3 = 423^{1019} (\text{mod }) 3337 =1410 \]
\[m_4 = 827^{1019} (\text{mod }) 3337 = 17\]
\[m_5 = 729^{1019} (\text{mod }) 3337 = 817\]
\[m_6 = 2066^{1019} (\text{mod }) 3337 = 13\]
\[m_7 = 3126^{1019} (\text{mod }) 3337 = 300\]
\[m_8 = 338^{1019} (\text{mod }) 3337 = 1800\]
\[m_9 = 109^{1019} (\text{mod }) 3337 = 1215\]
\[m_{10} = 2807^{1019} (\text{mod }) 3337 = 8\]
\[m_{11} = 537^{1019} (\text{mod }) 3337 = 308\]
\[m_{12} = 2112^{1019} (\text{mod }) 3337 = 1808\]
\[m_{13} =1752^{1019} (\text{mod }) 3337 = 1308\]
Berdasarkan kalkulasi diatas diperoleh plainteks
\[07 04 11 11 14 10 00 17 08 17 00 13 03 00 18 00 12 15 00 08 03 08 18 08 13 08\]
yang apabila disesuaikan dengan permisalan di atas diperoleh pesan berupa
\[\text{HELLOKARIRANDASAMPAIDISINI}\]
	\end{enumerate}
	\end{itemize}
\end{document}
