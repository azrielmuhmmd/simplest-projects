\documentclass[12pt,A4paper]{article}
\usepackage[top=1cm,right=2cm,bottom=3cm,left=2cm]{geometry}
\usepackage{amsmath}

\begin{document}
\begin{center}
Muhammad Rokan Azriel Prananta
\\ 24010121130045
\\ Tugas $1$ Mata Kuliah Aljabar Linear
\end{center}
\begin{enumerate}
\item Diberikan vektor-vektor basis
\[ \mathbf{v_1} = \left\lbrace 1,0,0 \right\rbrace, \mathbf{v_2} = \left\lbrace 2,2,0 \right\rbrace, \mathbf{v_3} = \left\lbrace 3,3,3 \right\rbrace \]
Misalkan terdapat vektor $\mathbf{x}=(x_1, x_2, x_3)$ dengan vektor-vektor basis di atas, maka untuk $\mathbf{x}= c_1\mathbf{v_1}+c_2\mathbf{v_2}+c_3\mathbf{v_3}$ untuk $c_n$ merupakan skalar, maka diperoleh notasi
\[ (x_1,x_2,x_3) = c_1(1,0,0) + c_2 (2,2,0) + c_3(3,3,3) \]
Selanjutnya diperoleh sistem persamaan
\[ \begin{cases}
c_1 + 2c_2 + 3c_3 = x_1 \\
2c_2 + 3c_3 = x_2 \\
c_3 = x_3
\end{cases}\]
Dari persamaan di atas diperoleh $c_3 = \frac{x_3}{3}$ sehingga
\[ 2c_2 + x_3 = x_2 \Leftrightarrow c_2 = \frac{x_2-x_3}{2} \]
\[ c_1 = x_21 - x_2 \]
Dari dua persamaan di atas diperoleh
\begin{eqnarray*}
(x_1,x_2,x_3) &=&
(x_1 - x_2) \mathbf{v_1} + \left(\frac{x_2-x_3}{2} \right) \mathbf{v_2} + \left( \frac{1}{x_3}\right) \mathbf{v_3} 
\end{eqnarray*}
Maka untuk  rumus $T(x_1,x_2,x_3)$ adalah
\begin{eqnarray*}
T(x_1,x_2,x_3) &=& (x_1 - x_2) (2,-1) + \left(\frac{x_2-x_3}{2} \right) (0,1) + \left( \frac{1}{x_3}\right) (5,3) \\
&=& (2x_1 - 2x_2, -x_1 + x_2) + \left(0,\frac{x_2-x_3}{2}  \right) + \left(\frac{5x_3}{3}, x_3 \right) \\
&=& \left(  2x_1 - 2x_2 +\frac{5x_3}{3}  , 
-x_1 + x_2 +  \frac{x_2-x_3}{2} + x_3\right)
\end{eqnarray*}
Maka untuk $T(-1,2,4)$ hasilnya adalah
\begin{eqnarray*}
T(-1,2,4) &=& \left(-2 - 4 +\frac{20}{3} , 
1 + 2 +  \frac{2-4}{2} + 4\right)\\
 &=& \left( \frac{2}{3}, 6 \right)
\end{eqnarray*}
\newpage
\item Mencari nilai eigen dan vektor eigen $ C = \begin{bmatrix} 
3 & -2 & 0 \\
-2 & 3 & 0 \\
0 &0 & 5
\end{bmatrix}$
\begin{enumerate}
    \item Mencari nilai eigen
\begin{eqnarray*}
\lambda I - C &=& \begin{bmatrix}
\lambda & 0 & 0 \\
0 & \lambda & 0 \\
0 & 0 & \lambda
\end{bmatrix} - 
\begin{bmatrix} 
3 & -2 & 0 \\
-2 & 3 & 0 \\
0 &0 & 5
\end{bmatrix} \\
&=& \begin{bmatrix}
\lambda - 3 & 2 & 0 \\
-2 & \lambda - 3 & 0 \\
0 & 0 & \lambda - 5
\end{bmatrix}
\end{eqnarray*}
Menentukan determinan dari hasil perhitungan matriks di atas yang disamadengankan dengan 0, maka
\begin{eqnarray*}
det (\lambda I - C) &=& \begin{vmatrix}
\lambda - 3 & 2 & 0 \\
-2 & \lambda - 3 & 0 \\
0 & 0 & \lambda - 5
\end{vmatrix} \\
0 &=& \begin{vmatrix}
\lambda - 3 & 2 & 0 \\
-2 & \lambda - 3 & 0 \\
0 & 0 & \lambda - 5
\end{vmatrix}
\end{eqnarray*}
Diperoleh nilai determinannya adalah
\begin{eqnarray*}
\left( (\lambda - 3) (\lambda - 3) ( \lembda - 5) \right) - \left( 4 (\lambda - 5) \right) &=& 0 \\
(\lambda - 5) \left( \lambda^2 - 6\lambda + 5 \right) &=& 0 \\
(\lambda - 5) (\lambda - 5) (\lambda - 1) &=& 0
\end{eqnarray*}
Diperoleh nilai eigen 
\[\lambda_1 = 5 \text{ atau } \lambda_2 = 1 \]
    \item Mencari vektor eigen
    \begin{itemize}
        \item Untuk $\lambda_1 = 5$
        \begin{eqnarray*}\begin{bmatrix}
2 & 2 & 0 \\
-2 & 2 & 0 \\
0 & 0 & 0
\end{bmatrix}  \begin{bmatrix}
x_1 \\
x_2 \\
x_3
\end{bmatrix} = \begin{bmatrix}
0 \\
 0\\
0
\end{bmatrix} \buildrel R_2 - R_1 \text{ dan } \frac{1}{2} \times R_1 \over  \implies \begin{bmatrix}
1 & 1 & 0 \\
0 & 0 & 0 \\
0 & 0 & 0
\end{bmatrix}  \begin{bmatrix}
x_1 \\
x_2 \\
x_3
\end{bmatrix} = \begin{bmatrix}
0 \\
 0\\
0
\end{bmatrix}
\end{eqnarray*}
Diperoleh persamaan
\[ x_1 = -x_2 \]
Misalkan $x_2 = a$ dan $x_3 = b$ $\implies x_1 = -a$ , maka vektor eigen yang sesuai untuk $\lambda = 5$ adalah
\[\mathbf{x} = \begin{bmatrix}
-a \\
a\\
b
\end{bmatrix} = 
\begin{bmatrix}
-a \\
a\\
0
\end{bmatrix} + \begin{bmatrix}
0 \\
0\\
b
\end{bmatrix} = a \begin{bmatrix}
-1 \\
1\\
0
\end{bmatrix} + b\begin{bmatrix}
0 \\
0\\
1
\end{bmatrix} \]
Untuk $a,b$ merupakan sembarang bilangan real dan bukan nol.
\item Untuk $\lambda = 1$
\[\begin{bmatrix}
-2 & 2 & 0 \\
-2 & -2 & 0 \\
0 & 0 & -4
\end{bmatrix}  \begin{bmatrix}
x_1 \\
x_2 \\
x_3
\end{bmatrix} = \begin{bmatrix}
0 \\
 0\\
0
\end{bmatrix} \buildrel R_2 + R_1 \text{ dan } \frac{1}{2} \times R_2 \over  \implies \begin{bmatrix}
0 & 0 & 0 \\
1 & -1 & 0 \\
0 & 0 & -4
\end{bmatrix}  \begin{bmatrix}
x_1 \\
x_2 \\
x_3
\end{bmatrix} = \begin{bmatrix}
0 \\
 0\\
0
\end{bmatrix}\]
\[ \buildrel  -\frac{1}{4} \times R_3 \over  \implies 
\begin{bmatrix}
0 & 0 & 0 \\
1 & -1 & 0 \\
0 & 0 & 1
\end{bmatrix}  \begin{bmatrix}
x_1 \\
x_2 \\
x_3
\end{bmatrix} = \begin{bmatrix}
0 \\
 0\\
0
\end{bmatrix}\]
Diperoleh persamaan
\[ \begin{cases}
x_1 = x_2 \\
x_3 = 0
\end{cases}\]
Misalkan $x_2 = a \implies x_1 = a$ , maka vektor eigen yang sesuai untuk $\lambda = 5$ adalah
\[\mathbf{x} = \begin{bmatrix}
a \\
a\\
0
\end{bmatrix} =  a
\begin{bmatrix}
1 \\
1\\
0
\end{bmatrix}  \]
Untuk $a$ merupakan sembarang bilangan real dan bukan nol.
    \end{itemize}
\end{enumerate}
\end{enumerate}
\end{document}
