\documentclass[12pt,A4paper]{article}
\usepackage[top=1cm,right=2cm,bottom=3cm,left=2cm]{geometry}
\usepackage{amsmath}
\usepackage{amssymb}
\begin{document}
\begin{center}
	Muhammad Rokan Azriel Prananta
	\\ 24010121130045
	\\ Tugas $1$ Mata Kuliah Aljabar II
	\\ Solusi Latihan Soal Nomor 1
\end{center}
\begin{enumerate}
	\item Ditinjau ring bilangan bulat $\left( \mathbb{Z}, +, \times\right)$ dan ring matriks $\left(M_2\left(\mathbb{Z}\right), +, \times\right)$ dengan
	\[ M_2(\mathbb{Z}) = \left\lbrace\begin{bmatrix}
		a & b \\
		b & a
	\end{bmatrix} | \forall a,b \in \mathbb{Z}\right\rbrace \]
	Didefinisikan fungsi $\varphi : M_2(\mathbb{Z}) \rightarrow \mathbb{Z}$ dengan
	\[ \varphi \left( \begin{bmatrix}
		a & b \\
		b & a
	\end{bmatrix}\right) = a - b, \forall \begin{bmatrix}
	a & b\\
	b & a
\end{bmatrix} \in M_2(\mathbb{Z})\]
\begin{enumerate}
	\item Bukti bahwa $\varphi$ merupakan suatu homomorfisma
	 \begin{enumerate}
	 	\item Akan ditunjukkan untuk operasi penjumlahan 
	 	\\ Diambil sembarang $\begin{bmatrix}
	 		a & b\\
	 		b & a
	 	\end{bmatrix}, \begin{bmatrix}
	 	c & d\\
	 	d & c
 	\end{bmatrix} \in M_2(\mathbb{Z})$ sedemikian sehingga
 	\begin{eqnarray*}
 		\varphi \left(\begin{bmatrix}
 			a & b\\
 			b & a
 		\end{bmatrix} + \begin{bmatrix}
 		c & d\\
 		d & c
 	\end{bmatrix}  \right) &=& \varphi \left(\begin{bmatrix}
 	a+c & b+d\\
 	b+d & a+c
 \end{bmatrix}\right) \\
	&=& \varphi\left((a + c) - (b + d)\right) \\
	&=& \varphi\left((a-b) + (c - d)\right) \\
	&=& \varphi \left(\begin{bmatrix}
		a & b\\
		b & a
	\end{bmatrix} + \begin{bmatrix}
		c & d\\
		d & c
	\end{bmatrix}  \right)
 	\end{eqnarray*}
 $\therefore$ Terbukti bahwa operasi penjumlahan pada pemetaan $\varphi$ dapat diawetkan.
 \item Akan ditunjukkan untuk operasi perkalian
 \\ Diambil sembarang $\begin{bmatrix}
 	a & b\\
 	b & a
 \end{bmatrix}, \begin{bmatrix}
 	c & d\\
 	d & c
 \end{bmatrix} \in M_2(\mathbb{Z})$ sedemikian sehingga
\begin{eqnarray*}
	\varphi \left(\begin{bmatrix}
		a & b\\
		b & a
	\end{bmatrix} \times \begin{bmatrix}
		c & d\\
		d & c
	\end{bmatrix}  \right) &=& \varphi \left(\begin{bmatrix}
		ac + bd & bc+ad\\
		bc+ad & bd+ac
	\end{bmatrix}\right) \\
	&=& \varphi\left((ac + bd) - (bc + ad)\right) \\
	&=& \varphi\left((ac - bc) + (bd - ad)\right) \\
	&=& \varphi\left((a - b)c + (b - a)d\right) \\
	&=& \varphi \left(\begin{bmatrix}
		ac & bc\\
		bc & ac
	\end{bmatrix} + \begin{bmatrix}
		bd & ad\\
		ad & bd
	\end{bmatrix}  \right) \\
	&=& \varphi \left(\begin{bmatrix}
		ac + bd & bc+ad\\
		bc+ad & bd+ac
	\end{bmatrix}\right) \\
&=& \varphi \left(\begin{bmatrix}
	a & b\\
	b & a
\end{bmatrix} \times \begin{bmatrix}
	c & d\\
	d & c
\end{bmatrix}  \right)
\end{eqnarray*}
$\therefore$ Terbukti bahwa operasi perkalian pada pemetaan $\varphi$ dapat diawetkan.
	 \end{enumerate}
 $\therefore$ Oleh karena i dan ii terbukti maka dapat disimpulkan bahwa pemetaan $\varphi$ merupakan suatu homomorfisma ring.
 \newpage
 \item Kernel dari $\varphi$
 \begin{eqnarray*}
 	Ker(\varphi) &=& \left\lbrace \begin{bmatrix}
 		a & b\\
 		b & a
 	\end{bmatrix} \in M_2(\mathbb{Z}) | \varphi\left(\begin{bmatrix}
 	a & b\\
 	b & a
 \end{bmatrix}\right) = 0 \right\rbrace \\
&=& \left\lbrace \begin{bmatrix}
	a & b\\
	b & a
\end{bmatrix} \in M_2(\mathbb{Z}) | a - b = 0 \right\rbrace \\
&=& \left\lbrace \begin{bmatrix}
	a & b\\
	b & a
\end{bmatrix} \in M_2(\mathbb{Z}) | a = b \right\rbrace \\
&=&\left\lbrace \begin{bmatrix}
	b & b\\
	b & b
\end{bmatrix} \in M_2(\mathbb{Z})  \right\rbrace \\
&=&\left\lbrace \begin{bmatrix}
	a & a\\
	a & a
\end{bmatrix} \in M_2(\mathbb{Z})  \right\rbrace
 \end{eqnarray*}
\item Bukti bahwa $M_2(\mathbb{Z})/Ker(\varphi) \cong \mathbb{Z}$
\\  Diketahui pemetaan 
\[\phi :M_2(\mathbb{Z})/Ker(\varphi) \rightarrow \mathbb{Z} \]
Diambil sembarang dua elemen yang bernilai sama, yaitu $\begin{bmatrix}
	a & b\\
	b & a
\end{bmatrix} + Ker(\varphi),\begin{bmatrix}
c & d\\
d & c
\end{bmatrix} + Ker(\varphi)  \in M_2(\mathbb{Z})/Ker(\varphi)$. Lalu akan ditunjukkan
\begin{enumerate}
	\item Pemetaan $\phi$ bersifat injektif
	\\ Oleh karena 
	\begin{equation}\tag{1}\begin{bmatrix}
		a & b\\
		b & a
	\end{bmatrix} + Ker(\varphi) = \begin{bmatrix}
		c & d\\
		d & c
	\end{bmatrix} + Ker(\varphi)
\end{equation}
,maka diperoleh $\varphi \left(\begin{bmatrix}
	a & b\\
	b & a
\end{bmatrix}\right) = \varphi \left(\begin{bmatrix}
c & d\\
d & c
\end{bmatrix}\right)$ $\Leftrightarrow \varphi \left(\begin{bmatrix}
a & b\\
b & a
\end{bmatrix}\right) - \varphi \left(\begin{bmatrix}
c & d\\
d & c
\end{bmatrix}\right) = 0'$ sedemikian sehingga $\begin{bmatrix}
a & b\\
b & a
\end{bmatrix} -\begin{bmatrix}
c & d\\
d & c
\end{bmatrix} \in  Ker(\varphi)$ yang menunjukkan bahwa persamaan 1 bersifat benar. 
\\ $\therefore$ Terbukti bahwa pemetaan $\phi$ bersifat injektif.
\item Pemetaan $\phi$ bersifat surjektif
\\ Diambil sembarang elemen $x \in \mathbb{Z}$, maka terdapat $\begin{bmatrix}
	a & b\\
	b & a
\end{bmatrix} \in M_2(\mathbb{Z})$ sehingga 
\[x = \varphi\left(\begin{bmatrix}
	a & b\\
	b & a
\end{bmatrix}\right)\]
Hal ini menunjukkan adanya $\begin{bmatrix}
	a & b\\
	b & a
\end{bmatrix} + Ker(\varphi) \in M_2(\mathbb{Z})/Ker(\varphi)$ sehingga memenuhi \[x = \varphi\left(\begin{bmatrix}
a & b\\
b & a
\end{bmatrix} + Ker(\varphi)\right)\]
$\therefore$ Hal ini menunjukkan bahwa pemetaan $\phi$ bersifat surjektif. 
\end{enumerate}
$\therefore$ Ole karena i dan ii terbukti maka dapat disimpulkan bahwa $M_2(\mathbb{Z})/Ker(\varphi) \cong \mathbb{Z}$
\end{enumerate}
\end{enumerate}
\end{document}
